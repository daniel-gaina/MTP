\documentclass{article}
\textwidth 130mm
\textheight 200mm

%************************ packages *********************************
\usepackage{booktabs} % for tabs
\usepackage{authblk}
\usepackage{amssymb}
\usepackage{amsmath}
\usepackage{amsthm}
\usepackage[inline]{enumitem}
\usepackage[colorlinks=true,linkcolor=blue]{hyperref}
\usepackage[latin1]{inputenc}
\usepackage{xy}
\xyoption{all}
\usepackage{tikz}
\usepackage{mathabx}
\usepackage{mathtools}

%************** line enumeration ********************************* 
\usepackage{xcolor}
%\usepackage{vruler} \setvruler[10pt][1][1][4][1][10pt][10pt][2pt][1.0\textheight]

%**************Theorem environments********************************* 
\renewenvironment{abstract}{\section*{Abstract}\small}{}
\newtheorem{definition}{Definition}
\newtheorem{example}[definition]{Example}
\newtheorem{application}[definition]{Application}
\newtheorem{theorem}[definition]{Theorem}
\newtheorem{exercise}[definition]{Exercise}
\newtheorem{lemma}[definition]{Lemma}
\newtheorem{proposition}[definition]{Proposition}
\newtheorem{corollary}[definition]{Corollary}
\newtheorem{fact}[definition]{Fact}
\newtheorem{remark}[definition]{Remark}
\newtheorem{framework}{Framework}[section]
\newtheorem{assumption}{Assumption}
\newtheorem{notation}[framework]{Notation}
%**************Keywords********************************* 
\def\keywords{\vspace{.5em}
{\noindent\textbf{Keywords}:\,\relax%
}}
\def\endkeywords{\par}
%************************ shortcuts ********************************
\newcommand{\PL}{{\mathbf{PL}}}
\newcommand{\HPL}{{\mathbf{HPL}}}

\newcommand{\FOL}{{\mathbf{FOL}}}
\newcommand{\HCL}{{\mathbf{HCL}}}
\newcommand{\REL}{{\mathbf{REL}}}
\newcommand{\HFOL}{{\mathbf{HFOL}}}
\newcommand{\HFOLS}{{\mathbf{HFOLS}}}
\newcommand{\HFOLR}{{\mathbf{HFOLR}}}
\newcommand{\HHCLS}{{\mathbf{HHCLS}}}

\newcommand{\OSA}{{\mathbf{OSA}}}
\newcommand{\HDOSA}{{\mathbf{HDOSA}}}


\newcommand{\Nom}{\mathtt{Nom}}
\newcommand{\Sig}{\mathtt{Sig}}
\newcommand{\Mod}{\mathtt{Mod}}
\newcommand{\Sen}{\mathtt{Sen}}

\newcommand{\Cat}{\mathtt{Cat}}
\newcommand{\Set}{\mathtt{Set}}

\newcommand{\I}{\mathtt{I}}
\newcommand{\HI}{\mathtt{HI}}

\newcommand{\N}{{\mathbb{N}}}

\newcommand{\E}{\mathcal{E}}
\renewcommand{\P}{\mathcal{P}}
\renewcommand{\L}{\mathcal{L}}

\newcommand{\C}{\mathtt{C}}


\newcommand{\F}{\mathtt{F}}
\newcommand{\K}{\mathtt{K}}

\newcommand{\A}{\mathfrak{A}}
\newcommand{\B}{\mathfrak{B}}
\newcommand{\D}{\mathfrak{D}}

\newcommand{\SP}{{\mathtt{SP}}}

\newcommand{\vdashv}{\mathrel{\models\!\!\!|}}
\newcommand{\red}{\!\upharpoonright\!}
%***************************************************************
\newcommand{\sort}{\mathtt{sort}}
\newcommand{\arity}{\mathtt{arity}}
\newcommand{\length}{\mathtt{length}}
\newcommand{\ari}{\mathtt{ar}}
\newcommand{\va}{\mathtt{va}}
\newcommand{\s}{\mathtt{s}}

\newcommand{\rigid}{\mathtt{r}}
\newcommand{\flexible}{\mathtt{f}}
\newcommand{\nom}{\mathtt{n}}
\newcommand{\act}{\mathfrak{a}}

\newcommand{\oS}{{\overline S}}
\newcommand{\oF}{{\overline F}}
\newcommand{\oP}{{\overline P}}
\newcommand{\oSigma}{{\overline\Sigma}}
\newcommand{\oDelta}{{\overline\Delta}}
\newcommand{\oX}{{\overline X}}
\newcommand{\ok}{{\overline k}}
\newcommand{\ow}{{\overline w}}
% ***************************************************************
\newcommand{\bigand}{\bigwedge}
\renewcommand{\implies}{\Rightarrow}
\newcommand{\at}[1]{@_{#1}\,}
\newcommand{\nec}[1]{[#1]}
\newcommand{\pos}[1]{\langle #1 \rangle}
\newcommand{\store}[1]{{\downarrow}#1\,{\cdot}\,}
\newcommand{\Forall}[1]{\forall #1\,{\cdot}\,}
\newcommand{\Exists}[1]{\exists #1\,{\cdot}\,}
\newcommand{\Q}[1]{\mathcal{Q}#1\,{\cdot}\,}
% ***************************************************************
\usepackage{mathpartir} %% for proof rules
\usepackage{xspace}

\newcommand{\rlImp}{(\textit{Imp})\xspace}
\newcommand{\rlMP}{(\textit{MP})\xspace}
\newcommand{\rlRet}{(\textit{Ret})\xspace}
\newcommand{\rlNEq}{(\textit{NEq})\xspace}
\newcommand{\rlNec}{(\textit{Nec})\xspace}
\newcommand{\rlSubst}{(\textit{Subst\({}_{s}\)})\xspace}

%***************************************************************

%%%%%%%%%%%%%%%%%%%%%%%%%%%%%%%%%%%%%%%%%%%%%%%%%%%%%%%%%%%%%%%%

%************************ title ********************************
\date{}
\title{Hybrid-dynamic Order-Sorted Algebra}
\author{}


%***************************************************************
\begin{document}
\maketitle

\section{Order-Sorted Algebra}

The order-sorted formalism that we consider in this paper, hereafter abbreviated $\OSA$, is a variation of the order-sorted algebra discussed in~\cite{Meseguer97} that supports both relational and equational atoms -- as opposed to only equational atoms.

\paragraph{Signatures}
The signatures are of the form \(\Sigma = (S, \leq, F, P)\), where 
\begin{enumerate}[label=\alph*)]

\item \((S, \leq)\) is a preorder, i.e.\ a set equipped with a reflexive and transitive relation, 

\item \((S, F, P)\) is a many-sorted first-order signature.

\end{enumerate}
Given a sort \(s \in S\), we denote by \([s]\) the set of its connected components $[s]=\{s'\in S\mid s'\equiv_\leq s\}$, where $\equiv_\leq$ is the equivalence on $S$ generated by the preorder $\leq$.
The signature $\Sigma$ is called \emph{sensible} if 
\begin{itemize}[nosep, leftmargin=1em]
\item for any operators $\sigma:\ari\to s,\sigma:\ari'\to s'\in F$ we have $\ari\equiv_\leq\ari'$ implies $s\equiv_\leq s'$.
\end{itemize}
The notion of sensible signature is a minimal syntactic requirement to avoid excessive ambiguity~\cite{Meseguer97}.
It is a much weaker requirement than preregularity~\cite{GoguenM92}.

\begin{assumption}
	For the sake of simplicity we assume that all connected components have a top sort.
By a slightly abuse of notation we let $[s]$ to denote both the connected component of $s$ and its top sort.
\end{assumption}
Throughout this paper, we let $\Sigma$, $\Sigma'$ and $\Sigma_i$ to range over $\OSA$ signatures of the form $(S,\leq,F,P)$, $(S',\leq',F',P')$ and $(S_i,\leq_i,F_i,P_i)$, respectively.

\paragraph{Signature morphisms.}
Signature morphisms $\varphi:(S,\leq,F,P)\to (S',\leq',F',P')$ are first-order signature morphisms $\varphi:(S,F,P)\to(S',F',P')$ such that 
\begin{enumerate}[label=\alph*),nosep,leftmargin=1.5em]

\item $\varphi:(S,\leq)\to(S',\leq')$ is monotonic, and

\item it preserves the sub-sort overloading: 
\begin{itemize}[nosep,leftmargin=1em]
\item for any operations $\sigma:\ari_1\to s_1, \sigma:\ari_2\to s_2\in F$ such that $\ari_1\equiv_\leq \ari_2$ we have  $\varphi_{\ari_1,s_1}(\sigma)=\varphi_{\ari_2,s_2}(\sigma)$, 
this means that operations with the same name (and in the same connected component) are mapped to operations with the same name, 

\item for any relations $\pi:\ari_1$ and $\pi:\ari_2$ such that $\ari_1\equiv_\leq \ari_2$ we have $\varphi_{\ari_1}(\pi)=\varphi_{\ari_2}(\pi)$, 
this means relations with the same name (and in the same connected component) are mapped to relations with the same name.
\end{itemize}
\end{enumerate}
We let $\Sig^\OSA$ to denote the category of $\OSA$ signature morphisms.

\begin{fact}
	For any non-sensible signature there exists an isomorphic sensible signature obtained by adding to the name of each operation symbol the connected components of their arities and sorts: 
	for example, an operation symbol $\sigma:\ari\to s$ will become $\sigma_{[\ari][s]}:\ari\to s$.
\end{fact}

\paragraph{Models}
Given a signature $\Sigma = (S, \leq, F, P)$, the $\Sigma$-models $\A$ interpret
\begin{enumerate}

\item each sort $s\in S$ as a set $\A_s$

\item each operation $(\sigma:\ari\to s)\in F$ as a function $\A_{\sigma:\ari\to s}\colon\A_\ari\to \A_s$,

\item each relation symbol $(\pi:\ari)\in P$ as a relation $\A_{\pi:\ari}\subseteq \A_\ari$, 

\end{enumerate}
such that:
\begin{enumerate}
 \item $\A_s\subseteq \A_{s'}$ whenever $s\leq s'$,

 \item $\A_{\sigma:\ari\to s}  $ and $ \A_{\sigma:\ari'\to s'}$ agree on $\A_\ari\cap \A_{\ari'}$ for all operations $\sigma:\ari\to s$ and $\sigma:\ari'\to s'$ in $F$ such that $[\ari]=[\ari']$, and
 \item $\A_{\pi:\ari}$ and $\A_{\pi:\ari'}$ coincide on \(\A_{\ari} \cap \A_{\ari'}\) for all relations $\pi:\ari$ and $\pi:\ari'$ in $P$ such that $[\ari]=[\ari']$.
\end{enumerate}
When there is no danger of confusion we may denote $\A_{\sigma:\ari\to s}$ and $\A_{\pi:\ari}$ simply by $\A_\sigma$ and $\A_\pi$, respectively.
A $\Sigma$-homomorphism $h:\A\to \B$ is a monotonic $S$-sorted function $h=\{h_s:\A_s\to \B_s\}_{s\in S}$ such that

\begin{enumerate}

\item if $s\leq s'$ then $h_s$ and $h_{s'}$ agree on $\A_s$,
  
\item $\A_\sigma ; h_s = h_\ari ; \B_\sigma$ for all $\sigma\colon\ari\to s\in F$, and

\item $h_\ari(\A_\pi)\subseteq \B_\pi$ for all $\pi\colon\ari \in P$.
\end{enumerate}
We let $\Mod^\OSA(\Sigma)$ to denote the category of $\Sigma$-models.

\paragraph{Terms}
The terms over a signature $\Sigma=(S,\leq,F,P)$ are defined inductively, as usual in the order-sorted-algebra literature.
For every sort $s \in S$, the set $T_{\Sigma, s}$ of $\Sigma$-terms is the least set such that:
\begin{itemize}
	\item $\sigma(t) \in T_{\Sigma, s}$ for all $\sigma \colon \ari \to s$ in $F$ and $t \in T_{\Sigma, \ari}$;
	
	\item $T_{\Sigma, s_0} \subseteq T_{\Sigma, s}$ whenever $s_0 \leq s$.
\end{itemize}
We denote by $T_{\Sigma, [s]}$ the set of all terms whose sort is in the same connected component as $s$; that is, $T_{\Sigma, [s]} = \bigcup \{ T_{\Sigma, s_0} \mid s_0\in S \mbox{ and } [s_0] = [s] \}$.

\paragraph{Sentences}
Let $\Sigma=(S,\leq,F,P)$ be a signature. 
There are three types of atomic sentences:
\begin{enumerate}
 \item \emph{equations} $t=t'$, where $t,t'\in T_{\Sigma,[s]}$ and $s\in S$,


 \item \emph{relations} $\pi(t)$, where $\pi:\ari\in P$ and $t\in T_{\Sigma,\ari}$.
\end{enumerate}
The set of $\Sigma$-sentences $\Sen^\OSA(\Sigma)$ are constructed from the above atomic sentences by applying Boolean connectives and quantification over finite sets of variables.
In order to avoid clashes of variables with constants from the target signatures when translating quantified sentences along signature morphisms we define variables for a signature $\Sigma$ as triples $(v,s,\Sigma)$, where 
 (a)~$v$ is the name of the variable, 
 (b)~$s$ is the sort of the variable, and
 (c) $\Sigma$ is the signature for which it was defined.
 Then we define quantified $\Sigma$-sentences as triples $\Q{X}\gamma$, where $\mathcal{Q}$ is a quantifier ($\forall$ or $\exists$), $X$ is a set of variables over $\Sigma$, and $\gamma$ is a sentence over $\Sigma[X]$ --- the signature obtained from $\Sigma$ by adding the variables in $X$ as constants to $\Sigma$.


If $\varphi\colon\Sigma\to \Sigma'$ and $\Forall{X}\gamma\in \Sen^\OSA(\Sigma)$ then $\varphi(\Forall{X}\gamma)=\Forall{X}\varphi'(\gamma)$, where $X'=\{(v,\varphi(s),\Sigma')\mid (v,s,\Sigma)\in X\}$ and $\varphi':\Sigma[X]\to\Sigma[X']$ maps every symbol in $\Sigma$ as $\varphi$ and any variable $(v,s,\Sigma)\in X$ to $(v,\varphi(s),\Sigma')\in X'$.
When there is no danger of confusion, we identify each variable by its name.
For more details about this topic, one may look into the quantification spaces defined in \cite{dia-qvh}.


\paragraph{Satisfaction relation.}
Given a signature $\Sigma=(S,\leq,F,P)$ and a $\Sigma$-model $\A$, the satisfaction of atomic sentences is based on the interpretation of terms:
\begin{enumerate}
\item $\A\models_\Sigma t=t'$ iff $\A_t= \A_{t'}$,

\item $\A\models\pi(t)$ iff $\A_t\in \A_\pi$.
\end{enumerate}
The satisfaction of sentences obtained by applying Boolean connectives and quantification is defined in the standard way.

\subsection{Congruences and quotients}

\begin{definition}[Order-sorted relation]
  Let \(\Sigma = (S, \leq, F, P)\) be an \(\OSA\)-signature, and \( \A \) a \(\Sigma\)-model.
  An $(S,\leq)$-relation on \(\A\) is an \(S\)-sorted relation \(\sim\) on \(\A\) such that \(\sim_{s}\) and \(\sim_{s_0}\) coincide on \(\A_{s} \cap \A_{s_0}\) for all sorts \(s, s_0 \in S\) such that \([s] = [s_0]\).

\end{definition}

\begin{definition}[Order-sorted congruence]
  An order-sorted $\Sigma$-congruence $\equiv$ on a \(\Sigma\)-model \(\A\) is an $(S,\leq)$-relation on $\A$ such that
  \begin{itemize}
    \item for all function symbols \(\sigma \colon \ari_{1} \to s_{1}\) and \(\sigma \colon \ari_{2} \to s_{2}\) with \([\ari_{1}] = [\ari_{2}]\), and elements \(a_{1}, a_{2} \in \A_{\ari_{1}} \cap \A_{\ari_{2}}\), if \(a_{1} \equiv_{\ari_{i}} a_{2}\) then \(\A_{\sigma}(a_{1}) \equiv_{s_{i}} \A_{\sigma}(a_{2})\);

    \item for all relation symbols \(\pi \colon \ari_{1}\) and \(\pi \colon \ari_{2}\) with \([\ari_{1}] = [\ari_{2}]\), and elements \(a_{1}, a_{2} \in \A_{\ari_{1}} \cap \A_{\ari_{2}}\), if \(a_{1} \equiv_{\ari_{i}} a_{2}\) and \(a_{1} \in \A_{\pi}\) then \(a_{2} \in \A_{\pi}\);
    \end{itemize}
\end{definition}

\begin{proposition}[Quotient]
  \label{prop:OSMA-quotient}
  Every \(\OSA\)-congruence \( \equiv\) on a model \( \A \) determines a \emph{quotient model} \(\widehat{\A}\), also denoted \(\A/_\equiv\), as follows:
  \begin{itemize}
  \item for every sort \(s \in S\), \(\widehat{\A}_{s} = \{\widehat{a} \in \widehat{\A}_{[s]} \mid a \in \A_{s}\}\), where \(\widehat{\A}_{[s]}\) is the quotient of the set \(\A_{[s]}\) determined by \(\equiv_{[s]}\);
  
  
  \item for every function symbol \(\sigma \colon \ari \to s\in F\) and \(a \in \A_{\ari}\), \(\widehat{\A}_{\sigma}(\widehat{a}) = \widehat{\A_{\sigma}(a)}\);
  
  \item for every relation \(\pi \colon \ari\in P\), \(\widehat{a} \in \widehat{\A}_{\pi}\) iff \(a \in \A_{\pi}\).
  
  \end{itemize}
\end{proposition}
%%%%%%%%%%%%%%%%%%%%%%%%%%%%%%%%%%%%%%%%%%%%%%%%%%%%%%%%%%%%%%%%%%%%%%%%%%%%%%%%%%%%
\section{Hybrid-Dynamic Order-Sorted Algebra}

Hybrid-Dynamic Order-Sorted Algebra with rigid symbols ($\HDOSA$) is defined based on the ideas used to define rigid first-order hybrid logic~\cite{DBLP:conf/wollic/BlackburnMMH19} and first-order hybrid logic with user-defined sharing \cite{dia-msc,dia-qvh}. 
The signatures contain a distinguished subset of ``rigid'' symbols such as sorts, operation and relation symbols which have the same interpretation in each world. 
This makes it possible to use a ``rigid'' semantics for the quantification, where the variables are interpreted uniformly across the worlds.

\paragraph{Signatures} 
\emph{The signatures} are of the form $\Delta=(\Sigma^n,\Sigma^r\subseteq \Sigma)$ where:
\begin{enumerate}
\item $\Sigma^\nom=(S^\nom,\leq^\nom,F^\nom,P^\nom)$ is an $\OSA$ signature of nominals with a single connected component,

\item $\Sigma^\rigid=(S^\rigid,\leq, F^\rigid,P^\rigid)$ is an $\OSA$ signature of rigid symbols, and

\item $\Sigma=(S,\leq,F,P)$ is an $\OSA$ signature. 
\end{enumerate}
We let $S^\flexible = S\setminus S^\rigid$ be set of flexible sorts, and $F^\flexible=F\setminus F^\rigid$ and $P^\flexible=P\setminus P^\rigid$ be the subsets of $F$ and $P$ that consist of flexible symbols (obtained by removing rigid symbols).
Throughout this paper we let $\Delta$, $\Delta'$ and $\Delta_i$ range over signatures of the form $(\Sigma^\nom,\Sigma^\rigid\subseteq\Sigma)$, $(\Sigma'^\nom,\Sigma'^\rigid\subseteq \Sigma')$ and $(\Sigma^\nom_i,\Sigma^\rigid_i\subseteq \Sigma_i)$, respectively.

\paragraph{Signature morphisms.}
The signature morphisms $\varphi:\Delta\to \Delta'$ are pairs $\varphi=(\varphi^n,\varphi)$, where:
\begin{enumerate}[leftmargin=1.5em,nosep,label=\alph*)]
 \item $\varphi^\nom:\Sigma^\nom\to\Sigma'^\nom$ is an $\OSA$  signature morphism mapping the top sorts of $\Sigma^n$ to the top sorts of $\Sigma'^\nom$, and

 \item $\varphi:\Sigma\to\Sigma'$ is an $\OSA$ signature morphism such that $\varphi(\Sigma^\rigid)\subseteq \Sigma'^\rigid$.
\end{enumerate}
%%%%%%%%%%%%%%%%%%%%%%%%%%%%%%%%%%%%%%%%%%%%%%%%%%%%%%%%%%%%%%%%%%%%%%%%%%%%%%%%%%%%
\paragraph{Kripke structures}
The $\Delta$-models are pairs $(W,M)$, where
\begin{enumerate}
\item $W\in|\Mod^\OSA(\Sigma^\nom)|$, and 
\item $M\colon W_{[s_1]}\times W_{[s_n]}\to|\Mod^\OSA(\Sigma)|$ is a mapping such that 
\begin{enumerate}
 \item $[s_1],\dots,[s_n]$ are all the top sorts of $\Sigma^\nom$,
 \item the rigid symbols have the same interpretations across the worlds, i.e. ${M_{w_1}\red_{\Sigma^\rigid}}$ and ${M_{w_2}\red_{\Sigma^\rigid}}$ are equal as order-sorted models, for all worlds $w_1,w_2\in|W|$; 
 this means that \(M_{w_{1}}\) and \(M_{w_{2}}\) have the same carrier sets for the rigid sorts, and the same interpretations of the rigid operation or predicate symbols.
 \end{enumerate}
\end{enumerate}

%%%%%%%%%%%%%%%%%%%%%%%%%%%%%%%%%%%%%%%%%%%%%%%%%%%%%%%%%%%%%%%%%%%%%%%%%%%%%%%%%%%%
\paragraph{Rigid terms} 
 Let $\Delta$ be a $\HDOSA$ signature such that $[s_1],\dots,[s_n]$ are all the top nominal sorts.
 The \emph{rigidification} of $\Sigma$ with respect to $\Sigma^\nom$ is the signature $\Sigma_@=(S_@,\leq_@,F_@,P_@)$, where 
 \begin{enumerate}
 \item $S_@=\{@_{\overline{k}} s \mid \overline{k}\in T_{\Sigma^\nom,[s_1]}\times\dots\times T_{\Sigma^\nom,[s_n]} \mbox{ and } s\in S\}$,
 
 \item $\leq_@=\{ (\at{\overline{k}}s_1,\at{\overline{k}}s_2 ) \mid \overline{k}\in T_{\Sigma^\nom,[s_1]}\times\dots\times T_{\Sigma^\nom,[s_n]} \mbox{ and } s_1\leq s_2  \}$,
 
 \item $F_@=\{@_{\overline{k}}\sigma\colon @_{\overline{k}}\ari \to @_{\overline{k}} s \mid \ok\in T_{\Sigma^\nom,[s_1]}\times\dots\times T_{\Sigma^\nom,[s_n]} \mbox{ and } \sigma\colon \ari\to s \in F \}$,~\footnote{$@_\ok (s_1\ldots s_n)=@_\ok s_1\ldots @_\ok s_n$ for all arities $s_1\ldots s_n$.}
 
 \item $P_@=\{@_\ok \pi\colon @_\ok \ari \mid \ok\in T_{\Sigma^\nom,[s_1]}\times\dots\times T_{\Sigma^\nom,[s_n]} \mbox{ and } \pi\colon\ari\in P\}$.
 \end{enumerate}
 Since the rigid symbols have the same interpretation across the worlds, we further define $@_\ok x=x$ for all tuples of nominals $\ok\in T_{\Sigma^\nom,[s_1]}\times\dots\times T_{\Sigma^\nom,[s_n]}$ and all symbols $x$ in $\Sigma^\rigid$.
 The set of \emph{rigid $\Delta$-terms} is $T_{\Sigma_@}$
 

 We define the interpretation of the rigid terms into Kripke structures:
 for any $\Delta$-model $(W,M)$, any possible world $\ow\in W_{s_1}\times\dots\times W_{s_n}$, and any rigid $\Delta$-term $t$, 
 \begin{enumerate}
  \item $M_{\ow,\sigma(t)} = (M_{\ow,\sigma})(M_{\ow,t})$, where $\sigma\colon\ari\to s\in F^\rigid$;
  \footnote{$M_{\ow,(t_1,\ldots,t_2)}=M_{\ow,t_1},\ldots,M_{\ow,{t_n}}$ for all lists of rigid terms $t_1,\ldots,t_n$.}
  
   \item $M_{\ow,(@_\ok \sigma)(t)} = (M_{\ow',\sigma}) (M_{\ow,t})$, where $(@_\ok \sigma\colon @_\ok \ari\to @_\ok s)\in F_@^\flexible$ and $\ow'=W_{\ok}$.~\footnote{$W_{(k_1,\dots,k_n)}=(W_{k_1},\dots,W_{k_n})$.}
 \end{enumerate}

%%%%%%%%%%%%%%%%%%%%%%%%%%%%%%%%%%%%%%%%%%%%%%%%%%%%%%%%%%%%%%%%%%%%%%%%%%%%%%%%%%%%
\paragraph{Sentences} 
Given a signature $\Delta$, the proper atomic $\Delta$-sentences consist of
 
\begin{enumerate}
 \item nominal equations $k_1=_{[s]} k_2$, where $k_1,k_2\in T_{\Sigma^\nom,[s]}$ and $[s]\in [S^\nom]$,

 \item nominal relations $\lambda(k_1,k_2)$, where $\lambda \colon s_1 s_2\in P^\nom$,  $k_1\in T_{\Sigma^\nom_@,s_1}$ and $k_1\in T_{\Sigma^\nom_@,s_2}$,

 \item rigid equations $t_1=_{[s]} t_2$, where $t_1,t_2\in T_{\Sigma_@,[s]}$ and $[s]\in [S_@]$, and

 \item rigid relations $\pi(t)$, where $\pi\colon\ari \in P_@$ and $t\in T_{\Sigma_@,\ari}$.
\end{enumerate}
 The set of $\Delta$-sentences is given by the following grammar:
$$e\Coloneqq k_1=k_2 \mid\lambda(k_1,k_2)\mid t_1=t_2 \mid \pi(t) \mid  \neg e\mid \vee E\mid \at{k} e\mid \pos{\lambda}e \mid \store{z}e'\mid \Exists{X,Y}e''$$
where 
\begin{enumerate*}[label=(\alph*)]
 \item $k_1=k_2$ is a nominal equation,
 \item $\lambda$ is a binary modality and $\lambda(k_1,k_2)$ is a nominal relation,
 \item $t_1=t_2$ is a rigid equation,
 \item $\pi(t)$ is a rigid relation,
 \item $E$ is a finite set of $\Delta$-sentences,
 \item $z$ is a nominal variable and $e'$ is a $\Delta[z]$-sentence,
 \item $X$ is a finite set of nominal variables,
       $Y$ is a finite set of variables of rigid sorts, and
       $e''$ is a $\Delta[X,Y]$-sentence.
\end{enumerate*}
%%%%%%%%%%%%%%%%%%%%%%%%%%%%%%%%%%%%%%%%%%%%%%%%%%%%%%%%%%%%%%%%%%%%%%%%%%%%%%%%%%%%
%%%%%%%%%%%%%%%%%%%%%%%%%%%%%%%%%%%%%%%%%%%%%%%%%%%%%%%%%%%%%%%%%%%%%%%%%%%%%%%%%%%%
%%%%%%%%%%%%%%%%%%%%%%%%%%%%%%%%%%%%%%%%%%%%%%%%%%%%%%%%%%%%%%%%%%%%%%%%%%%%%%%%%%%%
%%%%%%%%%%%%%%%%%%%%%%%%%%%%%%%%%%%%%%%%%%%%%%%%%%%%%%%%%%%%%%%%%%%%%%%%%%%%%%%%%%%%
%%%%%%%%%%%%%%%%%%%%%%%%%%%%%%%%%%%%%%%%%%%%%%%%%%%%%%%%%%%%%%%%%%%%%%%%%%%%%%%%%%%%
%%%%%%%%%%%%%%%%%%%%%%%%%%%%%%%%%%%%%%%%%%%%%%%%%%%%%%%%%%%%%%%%%%%%%%%%%%%%%%%%%%%%
%%%%%%%%%%%%%%%%%%%%%%%%%%%%%%%%%%%%%%%%%%%%%%%%%%%%%%%%%%%%%%%%%%%%%%%%%%%%%%%%%%%%
%%%%%%%%%%%%%%%%%%%%%%%%%%%%%%%%%%%%%%%%%%%%%%%%%%%%%%%%%%%%%%%%%%%%%%%%%%%%%%%%%%%%
%%%%%%%%%%%%%%%%%%%%%%%%%%%%%%%%%%%%%%%%%%%%%%%%%%%%%%%%%%%%%%%%%%%%%%%%%%%%%%%%%%%%
%%%%%%%%%%%%%%%%%%%%%%%%%%%%%%%%%%%%%%%%%%%%%%%%%%%%%%%%%%%%%%%%%%%%%%%%%%%%%%%%%%%%
%%%%%%%%%%%%%%%%%%%%%%%%%%%%%%%%%%%%%%%%%%%%%%%%%%%%%%%%%%%%%%%%%%%%%%%%%%%%%%%%%%%%
\bibliographystyle{abbrv}
\bibliography{hdosa}
%%%%%%%%%%%%%%%%%%%%%%%%%%%%%%%%%%%%%%%%%%%%%%%%%%%%%%%%%%%%%%%%%%%%%%%%%%%%%%%%%%%%
\end{document}
