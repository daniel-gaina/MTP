\documentclass{article}
\usepackage[left=2cm, right=2cm, top=2cm]{geometry}
 %*******************************************************
 \usepackage{booktabs} % For formal tables
%************************ packages *********************************
\usepackage{amssymb}
\usepackage{amsmath}
\usepackage{amsthm}
\usepackage{mathtools}
\usepackage[inline]{enumitem}
\usepackage[latin1]{inputenc}
\usepackage{xy}
\xyoption{all}
\usepackage{tikz}
\usepackage{xcolor}

\usepackage[skip=0pt]{caption}

%\usepackage{mathabx}

%************************ shortcuts ********************************
\newcommand{\PL}{{\mathsf{PL}}}
\newcommand{\HPL}{{\mathsf{HPL}}}
\newcommand{\HPLQ}{{\mathsf{HPLQ}}}

\newcommand{\FOL}{{\mathsf{FOL}}}
\newcommand{\HCL}{{\mathsf{HCL}}}
\newcommand{\REL}{{\mathsf{REL}}}
\newcommand{\HFOL}{{\mathsf{HFOL}}}
\newcommand{\HREL}{{\mathsf{HREL}}}


\newcommand{\HFOLS}{{\mathsf{HFOLS}}}
\newcommand{\HFOLR}{{\mathsf{HFOLR}}}
\newcommand{\HDFOL}{{\mathsf{HDFOL}}}

\newcommand{\POA}{{\mathsf{POA}}}
\newcommand{\POAR}{{\mathsf{POAR}}}
\newcommand{\HPOA}{{\mathsf{HPOA}}}
\newcommand{\HPOAS}{{\mathsf{HPOAS}}}
\newcommand{\HPOAR}{{\mathsf{HPOAR}}}

\newcommand{\Sig}{\mathsf{Sig}}
\newcommand{\Mod}{\mathsf{Mod}}
\newcommand{\Sen}{\mathsf{Sen}}

\newcommand{\SSb}{\mathsf{SSb}}
\newcommand{\SSt}{\mathsf{SSt}}
\newcommand{\Sb}{\mathsf{Sb}}
\newcommand{\St}{\mathsf{St}}
\newcommand{\Nom}{\mathsf{Nom}}
\newcommand{\Prop}{\mathsf{Prop}}
%*******************************************************
\newcommand{\Cat}{\mathbb{C}\mathsf{at}}
\newcommand{\Class}{\mathbb{C}\mathsf{lass}}
\newcommand{\Set}{\mathbb{S}\mathsf{et}}
%*******************************************************
\newcommand{\I}{\mathtt{I}}
\newcommand{\SI}{\mathtt{SI}}
%*******************************************************
\newcommand{\N}{{\mathbb{N}}}
\newcommand{\E}{\mathcal{E}}
\renewcommand{\P}{\mathcal{P}}
\newcommand{\V}{\mathcal{V}}
\newcommand{\X}{\mathsf{X}}
\newcommand{\Y}{\mathsf{Y}}
\newcommand{\Z}{\mathsf{Z}}
\newcommand{\Lis}{\mathsf{L}}
\newcommand{\C}{\mathsf{C}}
\newcommand{\Q}{\mathsf{Q}}
\newcommand{\D}{\mathsf{D}}
\newcommand{\F}{\mathsf{F}}
\newcommand{\K}{\mathsf{K}}
\newcommand{\A}{\mathcal{A}}
\newcommand{\B}{\mathsf{B}}
%*******************************************************
\newcommand{\Sub}{{\mathsf{Sub}}}
\newcommand{\ari}{{\mathsf{ar}}}
\newcommand{\nom}{{\mathsf{n}}}
\newcommand{\card}{{\mathsf{card}}}
\newcommand{\rigid}{{\mathtt{r}}}
\newcommand{\flexible}{{\mathtt{f}}}
\newcommand{\ext}{{\mathtt{e}}}
%*******************************************************
\newcommand{\LIST}{{\mathsf{LIST}}}
\newcommand{\List}{{\mathsf{List}}}
\newcommand{\Elt}{{\mathsf{Elt}}}
\newcommand{\TRIV}{{\mathsf{TRIV}}}
\newcommand{\BUFFER}{{\mathsf{BUFFER}}}
\newcommand{\SP}{{\mathsf{SP}}}

\newcommand{\err}{{\mathtt{err}}}
\newcommand{\Empty}{{\mathtt{empty}}}
\newcommand{\fifo}{{\mathtt{fifo}}}
\newcommand{\lifo}{{\mathtt{lifo}}}
\newcommand{\Read}{{\mathtt{read}}}
\newcommand{\del}{{\mathtt{del}}}
\newcommand{\shift}{{\mathtt{shift}}}
\newcommand{\lis}{{\mathtt{l}}}
\newcommand{\buf}{{\mathtt{b}}}
\newcommand{\ele}{{\mathtt{e}}}
\newcommand{\Ele}{{\mathtt{E}}}
\newcommand{\PR}{{\mathsf{PR}}}
\newcommand{\x}{{\mathfrak{x}}}
%*******************************************************
\usepackage{scalerel}

\newcommand{\bbsemicolon}{%
  \scalerel*{%
    \hbox{\usefont{U}{bbold}{m}{n} ;}%
  }{;}%
}

\newcommand{\comp}{\mathbin{\bbsemicolon}}
%*******************************************************
\newcommand{\modelsm}{\mathrel{\models\!\!\!|}}
\newcommand{\vdashv}{\mathrel{\vdash\!\!\dashv}}
\newcommand{\red}{\!\upharpoonright\!\!}
\newcommand{\ul}{\!\!\_}
\newcommand{\ur}{\_\!\!}
%*********sentence building operators********************
\newcommand{\At}[1]{\mathsf{at}_{#1}\, }
\newcommand{\at}[1]{@_{#1}\,}
\newcommand{\nec}[1]{[#1]}
\newcommand{\store}[1]{{\downarrow}#1\,{\cdot}\,}
\newcommand{\Forall}[1]{\forall #1\,{\cdot}\,}
\newcommand{\Exists}[1]{\exists #1\,{\cdot}\,}
\newcommand{\pos}[1]{{\langle}#1{\rangle}}
%*******************proof rules**************************
\newcommand{\inferrule}[2]{$\displaystyle\frac{#1}{#2}$}
%********************************************************
%% entailment properties  ........................................
\newcommand{\Union}{\textit{(Union)} }
\newcommand{\Monotonicity}{\textit{(Monotonicity)}}
\newcommand{\Transitivity}{\textit{(Transitivity)}}
\newcommand{\Translation}{\textit{(Translation)}}
%% first-order hybrid proof rules  ........................................
 \newcommand{\Cons}{$\mathit{(Cons)}$}
 \newcommand{\FalseE}{$\mathit{(False_E)}$}
 \newcommand{\FalseI}{$\mathit{(False_I)}$} 
 \newcommand{\NegE}{$\mathit{(Neg_E)}$}
 \newcommand{\NegI}{$\mathit{(Neg_I)}$}
 \newcommand{\NegD}{$\mathit{(Neg_D)}$} 
 \newcommand{\DisjE}{$\mathit{(Disj_E)}$}
 \newcommand{\DisjI}{$\mathit{(Disj_I)}$} 
  \newcommand{\ConjE}{$\mathit{(Conj_E)}$}
 \newcommand{\ConjI}{$\mathit{(Conj_I)}$} 
 \newcommand{\PosI}{$\mathit{(Pos_I)}$}
 \newcommand{\PosE}{$\mathit{(Pos_E)}$} 
 \newcommand{\QuantI}{$\mathit{(Quant_I)}$}
 \newcommand{\QuantE}{$\mathit{(Quant_E)}$} 
 \newcommand{\Subst}{$\mathit{(Subst)}$}
 \newcommand{\StoreI}{$\mathit{(Store_I)}$}
 \newcommand{\StoreE}{$\mathit{(Store_E)}$}  
 \newcommand{\RetI}{$\mathit{(Ret_I)}$}
 \newcommand{\RetE}{$\mathit{(Ret_E)}$}  
 \newcommand{\RetD}{$\mathit{(Ret_D)}$}
 \newcommand{\RetH}{$\mathit{(Ret_H)}$}
 %% HPL  ........................................
 %\newcommand{\Reflexivity}{\textit{(Reflexivity)}} 
 %\newcommand{\Congruence}{\textit{(Congruence)}}
 %\newcommand{\Coherence}{\textit{(Coherence)}}
%% HFOLS  ........................................
 \newcommand{\Rnom}{$\mathit{(R^n)}$}
 \newcommand{\Pnom}{$\mathit{(P^n)}$}
 \newcommand{\Cnom}{$\mathit{(C^n)}$}
 \newcommand{\Whyb}{$\mathit{(W^h)}$}
 \newcommand{\Wrig}{$\mathit{(W^r)}$}
 \newcommand{\Rhyb}{$\mathit{(R^h)}$}
 \newcommand{\Shyb}{$\mathit{(S^h)}$}
 \newcommand{\Thyb}{$\mathit{(T^h)}$}
 \newcommand{\Fhyb}{$\mathit{(F^h)}$}
 \newcommand{\Frig}{$\mathit{(F^r)}$}
 \newcommand{\Ffle}{$\mathit{(F^f)}$}
 \newcommand{\Phyb}{$\mathit{(P^h)}$}
 \newcommand{\Rintro}{$\mathit{(R^I)}$}
 \newcommand{\Relim}{$\mathit{(R^E)}$}
%%% .................................................... 
 \newcommand{\Ref}{$\mathit{(Ref)}$}
 \newcommand{\Sym}{$\mathit{(Sym)}$}
 \newcommand{\Rew}{$\mathit{(Rew)}$}
 \newcommand{\Trans}{$\mathit{(Trans)}$}
%*******************spacing******************************
\newcommand{\Space}{\hspace{4mm}}
%******************figure label*************************************
\newcounter{nr}
\newcounter{nrf}
\newcommand{\figlabel}[2]{{\refstepcounter{nrf}\label{#1}\textsf{\small Fig. \arabic{nrf}. #2}}}

 \newcommand{\myfigure}[3]{
  \setcounter{nrf}{\value{figure}}
  \medskip
  \begin{minipage}{\textwidth}
   \begin{center}
    
    #3
    
    \smallskip
    
    \figlabel{#1}{#2}
    
    \medskip
    
  \end{center} 
 \end{minipage} 
 \setcounter{figure}{\value{nrf}}
 }

\newcounter{nrt}
\newcommand{\tablabel}[2]{{\refstepcounter{nrt}\label{#1}\textsf{\small Table \arabic{nrt}. #2}}}

\newcommand{\mytable}[3]{
  \setcounter{nrt}{\value{table}}
  \medskip
  \begin{minipage}{\textwidth}
   
   \begin{center}
    
    #3
    
    \smallskip
    
    \tablabel{#1}{#2}
    
    \medskip
    
  \end{center} 
  
 \end{minipage} 
 \setcounter{table}{\value{nrt}}
 }

%**************proof environments************************

\usepackage{calc}

\usepackage{longtable}

\newcommand{\tabStretch}[1]{\renewcommand{\arraystretch}{#1}}
\newcommand{\tabColSep}[1]{\setlength{\tabcolsep}{#1}}
\usepackage{array,etoolbox}

%% BEGIN ProofSteps  %%%%%%%%%%%%%%%%%%%%%%%%%%%%%%%%%%%%%%%%%%%%%%%%%%%%%%%%%%%
\makeatletter

\newlength{\PS@lastparam}
\newlength{\PSlastparam}
\newcommand{\PSlp}{%
  \setlength{\PSlastparam}{\PS@lastparam}%
  \the\PSlastparam
}
\def\PS@sub@lastparam{}

\newcommand{\PS@numwidth}{99}
\newcommand{\PSnumwidth}[1]{%
  \renewcommand{\PS@numwidth}{#1}%
}

\newcommand{\PS@style}{\small}
\newcommand{\PS@numstyle}{\footnotesize}

\newlength{\PSindent}
\setlength{\PSindent}{0pt}
\newlength{\PS@extraindent}
\setlength{\PS@extraindent}{0pt}
\newlength{\PSpre}
\setlength{\PSpre}{\smallskipamount}
\newlength{\PSpost}
\setlength{\PSpost}{\smallskipamount}

\newlength{\PS@Nwidth}
\newlength{\PS@Swidth}
\newlength{\PS@Ewidth}
\newlength{\PScolsep}
\setlength{\PScolsep}{1.5em}

\newcommand{\PS@rownumber}{%
  \ifPS@subsubsteps
  \thePSsubstepc.%
  \the\numexpr\value{PSsubsubstepc}+1\relax
  \else
  \ifPS@substeps
  \thePSstepc.%
  \the\numexpr\value{PSsubstepc}+1\relax
  \else
  \the\numexpr\value{PSstepc}+1\relax
  \fi\fi
}
\newcommand{\PS@step}{%
  \ifPS@subsubsteps
  \refstepcounter{PSsubsubstepc}%
  \else
  \ifPS@substeps
  \refstepcounter{PSsubstepc}%
  \else
  \refstepcounter{PSstepc}
  \fi\fi%
}

\newif\ifPS@inprogress
\newif\ifPS@substeps
\newif\ifPS@subsubsteps
\newif\ifPS@continued
\newif\ifPS@subcontinued
\newcounter{PSc}
\newcounter{PSstepc}[PSc]
\newcounter{PSsubstepc}[PSstepc]
\renewcommand{\thePSsubstepc}{\thePSstepc.\arabic{PSsubstepc}}
\newcounter{PSsubsubstepc}[PSsubstepc]
\renewcommand{\thePSsubsubstepc}{\thePSsubstepc.\arabic{PSsubsubstepc}}

\newenvironment{proofsteps}[1]{%
  \global\settowidth{\PS@lastparam}{\PS@style\hspace*{#1}}
  \ifPS@continued\else\refstepcounter{PSc}\fi
  \begingroup
  \setlength{\LTpre}{\PSpre}%
  \setlength{\LTpost}{\PSpost}%
  \renewcommand{\arraystretch}{1.2}
  \setlength{\tabcolsep}{0pt}
  \noindent\PS@style
  \settowidth{\PS@Nwidth}{\PS@numstyle\PS@numwidth}%
  \setlength{\PS@Swidth}{#1}%
  \addtolength{\PS@Swidth}{-\PS@extraindent}%
  \setlength{\PS@Ewidth}{\linewidth}%
  \addtolength{\PS@Ewidth}{-\PSindent}%
  \addtolength{\PS@Ewidth}{-\PS@extraindent}%
  \addtolength{\PS@Ewidth}{-\PS@Nwidth}%
  \addtolength{\PS@Ewidth}{-\PScolsep}%
  \addtolength{\PS@Ewidth}{-\PS@Swidth}%
  \addtolength{\PS@Ewidth}{-\PScolsep}%
  \PS@inprogresstrue
  \longtable{%
    @{\hspace*{\PSindent}\hspace*{\PS@extraindent}\makebox[\PS@Nwidth][r]{\PS@rownumber}}%
    @{\hskip\PScolsep}>{\PS@step}p{\PS@Swidth}%
    @{\hskip\PScolsep}>{\footnotesize\raggedright\arraybackslash}p{\PS@Ewidth}%
  }%
}{%
  \ifPS@inprogress
  \addtocounter{table}{-1}%
  \endlongtable  
  \endgroup
  \PS@continuedfalse
  \PS@inprogressfalse
  \else\fi
}

\newcommand{\PSbreak}[1]{%
  \endproofsteps
  \par\medskip
  #1
  \medskip\par
  \PS@continuedtrue
  \proofsteps{\PS@lastparam}%
}

\newif\ifPS@sub@inprogress
\newenvironment{substeps}[1]{%
  \global\def\PS@sub@lastparam{#1}%
  \endproofsteps
  \settowidth{\PS@extraindent}{\PS@numstyle\PS@numwidth}%
  \addtolength{\PS@extraindent}{\PScolsep}%
  \addtolength{\PS@extraindent}{\widthof{\PS@numstyle.#1}}%
  \PS@substepstrue
  \PS@continuedtrue
  \PS@sub@inprogresstrue
  \proofsteps{\PS@lastparam}%
}{%
  \ifPS@sub@inprogress
    \endproofsteps
    \ifPS@laststep
      \PS@laststepfalse
      \PS@inprogressfalse
    \else
      \setlength{\PS@extraindent}{0pt}%
      \PS@substepsfalse
      \PS@continuedtrue
      \proofsteps{\PS@lastparam}%
    \fi
    \PS@sub@inprogressfalse
  \else
    \ifPS@laststep
      \PS@laststepfalse
      \PS@inprogressfalse
    \else
      \setlength{\PS@extraindent}{0pt}%
      \PS@substepsfalse
      \PS@continuedtrue
      \proofsteps{\PS@lastparam}%
    \fi
  \fi
}

\newif\ifPS@laststep
\newcommand{\laststep}{\global\PS@laststeptrue}
\newif\ifPS@lastsubstep
\newcommand{\lastsubstep}{\global\PS@lastsubsteptrue}

\newenvironment{subsubsteps}[1]{%
  \endproofsteps
  \addtolength{\PS@extraindent}{\widthof{\PS@numstyle\PS@numwidth}}%
  \addtolength{\PS@extraindent}{\PScolsep}%
  \addtolength{\PS@extraindent}{\widthof{\PS@numstyle.\PS@sub@lastparam}}%
  \addtolength{\PS@extraindent}{\widthof{\PS@numstyle.#1}}%
  \PS@subsubstepstrue
  \PS@continuedtrue
  \proofsteps{\PS@lastparam}%
}{
  \endproofsteps
  \ifPS@lastsubstep
    \PS@lastsubstepfalse
    \PS@sub@inprogressfalse
    \PS@subsubstepsfalse
  \else
    \settowidth{\PS@extraindent}{\PS@numstyle\PS@numwidth}%
    \addtolength{\PS@extraindent}{\PScolsep}%
    \addtolength{\PS@extraindent}{\widthof{\PS@numstyle.\PS@sub@lastparam}}%
    \PS@subsubstepsfalse
    \PS@continuedtrue
    \proofsteps{\PS@lastparam}%
  \fi
}

\newcommand{\adjustcol}[1]{%
  \global\advance\@colroom-#1%
}

\makeatother
%% END ProofSteps  %%%%%%%%%%%%%%%%%%%%%%%%%%%%%%%%%%%%%%%%%%%%%%%%%%%%%%%%%%%%%

% *******************************************************************
\newcommand{\psqed}{%
  \vspace{-\baselineskip}\vspace{-1\smallskipamount}
}

\newcommand*{\pcformat}[1]{%
  [\;{\normalfont\itshape #1}\;]%
}

\newenvironment{proofcases}[1][]{%
  \description[font=\pcformat, leftmargin=\parindent, #1]%
}{\enddescription}
\newcommand{\proofcase}[1]{%
\item[\textnormal{\textsf{[#1]}}]%
}
%**************Theorem environments********************************* 
\newtheorem{framework}{Framework}[section]
\newtheorem{assumption}{Assumption}
\newtheorem{notation}{Notation}
\newtheorem{remark}{Remark}[section]
\newtheorem{fact}{Fact}[section]
\newtheorem{example}{Example}
\newtheorem{lemma}{Lemma}
%**************Theorem environments********************************* 
% ***************************************************************

\newcommand{\minisec}[1]{%
  \par\addvspace{\smallskipamount}\noindent%
  \textit{#1}.\enspace%
}
%*******************************************************
\begin{document}

\section{Plastic buffer} \label{section:case:study}

In this section, we describe briefly 
\begin{enumerate*}
\item  \emph{the specification method}, and 
\item \emph{the verification technique} 
\end{enumerate*}
we intend to build on top of the foundation developed in the present contribution. 


\subsection{Specification method}

We envision a specification methodology where the rigid data types are built outside the hybrid specification.
For example, a hybrid specification in $\HFOLR$ has a signature $(\Nom,\Lambda,\Sigma^\rigid\subseteq\Sigma)$, where $\Sigma^\rigid=(S^\rigid,F^\rigid,P^\rigid)$ and $\Sigma=(S,F,P)$.
Practitioners will start by specifying the rigid data types, i.e. a first-order specification with the signature $\Sigma^\rigid$. 
This is followed by the definition of 
\begin{enumerate*}[label=(\alph*)]
\item nominals, 
\item accessibility relations between states, and 
\item flexible data types,
\end{enumerate*} 
 in such a way that  no `junk' and no `confusion' are added to the rigid data types, i.e. the $\Sigma^\rigid$-models previously defined are preserved.
 For the sake of simplicity, in practice, a variable is identified only by its name; by a slight abuse of notation, for each inclusion $\chi\colon\Delta\hookrightarrow\Delta'$ and any $\Delta$-sentence $\gamma$, we let $\gamma$ denote $\chi(\gamma)$.
 
 In this paper, only basic specifications $\SP$ are considered, that is $\SP=(\Sig(\SP),\Sen(\SP))$, where $\Sig(\SP)$ is a signature and $\Sen(\SP)$ is a set of sentences over $\Sig(\SP)$.

\begin{example} \label{ex:list}
We define the following specification of lists in $\FOL$:

\begin{itemize}
\item [] 
 
 spec $\LIST$
 
 sorts $\Elt\ \List $ 
 
 op $\err\colon \to \Elt$ 
 
 op $\Empty \colon \to \List$ 
 
 op  $\_\comp\_ \colon \List \Elt \to \List$ 
 
 vars $\Lis\ \Q :\List$ 
 
 var $\F : \Elt$ 
   
 eq-1 $\Forall{\Lis}\Lis \comp \err = \Lis$
 
 eq-2 $\Forall{\Lis} \Lis = \Empty \vee \Exists{\Q, \F } \Lis = \Q \comp \F$

\end{itemize}

\psqed\qed\end{example}
 
 The constant $\err$ can be regarded as an error element which can be used to describe partial functions on lists. 
 According to the first sentence, $\err$ should not be regarded as an element of any list.
 By the second sentence, a list is either empty or it is obtained from another list by adding one element.
 The specification $\LIST$ provides the rigid data types for the hybrid specification presented next.
 

\begin{example} \label{ex:buffer}
The hybrid specification $\BUFFER$ defined below consists of a buffer with two distinct operation modes:
\begin{enumerate*}[label=(\alph*)]
\item `lifo', where it behaves as a stack, and
\item `fifo', where it behaves as a queue.
\end{enumerate*}
The alternation of configurations is triggered by an event `shift'.

 \begin{itemize}
 \item[]
 
 spec $\BUFFER[\LIST]$
 
 nominals $\lifo\ \fifo$
 
 modality $\shift:2$
 
 op $\Read : \List \to \Elt$
 
 op $\del : \List \to \List$ 
 
 vars $\Ele\ \F: \Elt$
 
 var $\Lis:\List$
 
 rel-1 $\at{\fifo}\underline{\shift}(\lifo)$
   
 rel-2 $\at{\lifo}\underline{\shift}(\fifo)$ 
   
 eq-3 $\Read(\Empty) = \err$
   
 eq-4 $\Forall{ \Lis,\Ele} (@_\lifo\Read)(\Lis \comp \Ele) =  \Ele$
   
 eq-5 $\Forall{\Ele}(@_\fifo\Read)(\Empty \comp \Ele) =  \Ele$
   
 eq-6 $\Forall{\Lis,\Ele,\F}(@_\fifo\Read)(\Lis \comp \Ele \comp \F) =  (@_\fifo\Read)(\Lis \comp \Ele)$
   
 eq-7 $\del(\Empty) = \Empty$
   
 eq-8 $\Forall{\Lis,\Ele}(@_\lifo\del)(\Lis \comp \Ele) =  \Lis$
   
 eq-9 $\Forall{\Ele}(@_\fifo\del)(\Empty \comp \Ele) =  \Empty$
   
 eq-10 $\Forall{\Lis,\Ele,\F} (@_\fifo\del)(\Lis \comp \Ele \comp \F) =  (@_\fifo\del)(\Lis \comp \Ele) \comp \F$
 \end{itemize}
\psqed\qed\end{example}
  
 The $\REL$ component of the hybrid signature consists of two nominals $\fifo$ and $\lifo$ and one binary modality $\shift$. 
 The signature of rigid symbols is the signature of $\LIST$.
 There are two flexible operation symbols, $\del$ and $\Read$.
  
 The system has two operation modes, $\lifo$, when it behaves like a stack, and $\fifo$, when it behaves like a queue. 
 The binary modality $\shift$ makes the transition between $\lifo$ and $\fifo$ modes according to the nominal relations $\at{\fifo}\underline{\shift}(\lifo)$ and $\at{\lifo}\underline{\shift}(\fifo)$.
 The function symbol $\Read$ denotes the operation which returns the top/front of a stack/queue, and $\del$ denotes the operation `pop'. 
 Notice that $\Read$ and $\del$ play different roles in each operation mode.
 
 The models of $\BUFFER$ consists of all Kripke structures $(W,M)$ over the signature of $\BUFFER$ which 
 \begin{enumerate*}[label=(\alph*)]
  \item have a rigid $\LIST$ structure, that is $M_\lifo\red_{\Sig(\LIST)}=M_\fifo\red_{\Sig(\LIST)}$ is a model of $\LIST$, and 
  
  \item satisfy the axioms defined in Example~\ref{ex:buffer}.
\end{enumerate*}
 This construction is particularly useful for structured specifications which are obtained from basic specifications by applying specification building operators such as union, translation, hiding or freeness.
 As for Example~\ref{ex:buffer}, notice that any Kripke structure over $\Sig(\BUFFER)$ which satisfies $\Sen(\BUFFER)$ (i.e. all sentences defined in Example~\ref{ex:list} and Example~\ref{ex:buffer}) is a model of $\BUFFER$.

%%%%%%%%%%%%%%%%%%%%%%%%%%%%%%%%%%%%%%%%%%%%%%%%%%%%%%%%%%%%%%%%%%%%%%%%%%%%%

\subsection{Formal verification}

 This section is dedicated to proving that  $\BUFFER$ satisfies the following property:
 $$ \Forall{\Lis}(@_\lifo\del)((@_\fifo\del)(\Lis))=(@_\fifo\del)((@_\lifo\del)(\Lis))$$
 This means that the order of deleting the front and the top element from a list is irrelevant w.r.t. the final result.
 In order to implement efficient proof strategies, one often needs to derive new proof rules from the original ones.

%%%%%%%%%%%%%%%%%%%%%%%%%%%%%%%%%%%%%%%%%%%%%%%%%%%%%%%%%%%%%%%%%%%%%%%%%%%%%

\mytable{tab:derived-rules}{Derived proof rules for $\HFOLR$}{
\small
\begin{tabular}{|l l|} 
\hline
 & \\ 
 \Ref~\inferrule{}{\Gamma\vdash \Forall{X}t = t} &
 \Sym~\inferrule{\Gamma\vdash \Forall{X}t_1=t_2}{\Gamma\vdash \Forall{X}t_2 = t_1 } \\
 
  & \\
  
  \Trans~\inferrule{\Gamma\vdash \Forall{X} t_1 = t_2 \Space \Gamma\vdash\Forall{X} t_2 = t_3}{\Gamma\vdash \Forall{X}t_1=t_3} &
  \Rew~\inferrule{\Gamma\vdash \Forall{X}t_1 = t_2} {\Gamma\vdash \Forall{Y} t[\theta(t_1)_p] = t[\theta(t_2)]_p} 
 [ \ $\theta\colon X\to T_{@\Sigma}(Y)$ \ ] \\
 & \\ 
\hline
\end{tabular}
}

%%%%%%%%%%%%%%%%%%%%%%%%%%%%%%%%%%%%%%%%%%%%%%%%%%%%%%%%%%%%%%%%%%%%%%%%%%%%% 

  Since the present contribution is not dedicated to the presentation of a formal method, minimally, we define a system of proof rules in Table~\ref{tab:derived-rules}, which allows one to avoid complex formal proofs for obvious  properties.
  Notice that $e[t_1\leftarrow t_2]$ is the sentence obtained from $e$ by substituting $t_2$ for $t_1$, while  $t|_p$ is the subterm of $t$ at position $p$ and $t[\theta(t_i)]_p$ is the term obtained from $t$ by substituting $\theta(t_i)$ for $t|_p$ at position $p$. 

 For the sake of simplicity, we denote 
 by $\Delta$ the signature $\Sig(\BUFFER)$,
 by $\Gamma$ the set of sentences $\Sen(\BUFFER)$, and
 by $\mathtt{\PR(\Lis)}$ the formula $(@_\lifo\del)((@_\fifo\del)(\Lis))=(@_\fifo\del)((@_\lifo\del)(\Lis))$.

\begin{lemma} 
 We assume that the variable $\Lis$ is of sort $\List$, and the variables $\Ele$ and $\F$ are of sort $\Elt$.
 \label{lemma:buffer} \

\begin{enumerate}
\item  \label{lemma:buffer-1} $\Gamma\vdash \PR(\Empty)$;

\item  \label{lemma:buffer-2} $\Gamma\vdash \Forall{\Ele}\PR(\Empty\comp\Ele)$;

\item  \label{lemma:buffer-3} $\Gamma\vdash \Forall{\Lis,\Ele,\F}\PR(\Lis \comp \Ele \comp \F)$;

\item \label{lemma:buffer-4} $\Gamma\vdash \Forall{\Lis,\Ele}\PR(\Lis \comp \Ele)$;

\item \label{lemma:buffer-4} $\Gamma\vdash \Forall{\Lis}\PR(\Lis)$. 
\end{enumerate}
\end{lemma}
\begin{proof}

The first two assertions are straightforward to prove.
We start with the third assertion.
 \begin{proofsteps}{32em}
  
  \label{ps:buffer-3-1}
  $\Gamma\vdash \Forall{\Lis,\Ele,\F}(@_\lifo\del)((@_\fifo\del)(\Lis\comp\Ele\comp\F))=
  (@_\lifo\del)((@_\fifo\del)(\Lis\comp\Ele)\comp\F)$  & 
  by \Rew~ from eq-10\\
  
  \label{ps:buffer-3-2}
  $\Gamma\vdash\Forall{\Lis,\Ele,\F} (@_\lifo\del)((@_\fifo\del)(\Lis\comp\Ele)\comp\F) = 
  (@_\fifo\del)(\Lis\comp\Ele)$ &
  by \Rew~from eq-8\\ 
  
  \label{ps:buffer-3-3}
  $\Gamma\vdash\Forall{\Lis,\Ele,\F} (@_\fifo\del)((@_\lifo\del)(\Lis\comp\Ele\comp\F))= 
  (@_\fifo\del)(\Lis\comp\Ele)$ & 
  by \Rew~ from eq-8 \\
  
  \label{ps:buffer-3-4}
  $\Gamma\vdash\Forall{\Lis,\Ele,\F} (@_\fifo\del)(\Lis\comp\Ele) =
  (@_\fifo\del)((@_\lifo\del)(\Lis\comp\Ele\comp\F))$ & 
  by \Sym~ from  \ref{ps:buffer-3-3}\\
  
  $\Gamma\vdash\Forall{\Lis,\Ele,\F}(@_\lifo\del)((@_\fifo\del)(\Lis\comp\Ele\comp\F))=
  (@_\fifo\del)((@_\lifo\del)(\Lis\comp\Ele\comp\F))$ & 
  by \Trans~ from \ref{ps:buffer-3-1}, \ref{ps:buffer-3-2} and \ref{ps:buffer-3-4}
 \end{proofsteps}
 
 We prove the fourth assertion:
 \begin{proofsteps}{21em}
   
  \label{ps:buffer-4-1} $\Gamma\vdash_{\Delta[\Lis,\Ele]} \Lis = \Empty \vee \Exists{\Q,\F}  \Lis=\Q\comp\F$ & 
  from eq-2, 
  \newline since eq-2  is a rigid sentence\\
 
  \label{ps:buffer-4-2} $\Gamma\cup \{ \Lis = \Empty\} \vdash_{\Delta[\Lis,\Ele]} \PR[\Lis\comp\Ele]$ & 
   \substeps{9}
   $\Gamma\vdash_{\Delta[\Lis,\Ele]} \PR[\Empty\comp\Ele] $  &
   by \Rew~ from the second assertion \\
   
   \label{ps:buffer-22} 
   $\Gamma\cup \{ \Lis = \Empty\} \vdash_{\Delta[\Lis,\Ele]} \PR[\Empty;\Ele]$ &
   by \Transitivity~ and \Monotonicity\\ 
   
   $\Gamma\cup \{ \Lis = \Empty\} \vdash_{\Delta[\Lis,\Ele]} \PR[\Lis\comp\Ele]$ &
   by \Rew~ and \Trans~ from \ref{ps:buffer-22}
   \endsubsteps
  \label{ps:buffer-4-3} 
  $\Gamma\cup \{ \Exists{\Q,\F}\Lis = \Q\comp\F \} \vdash_{\Delta[\Lis,\Ele]} \PR[\Lis\comp\Ele]$ & 
  \substeps{9}
    \label{ps:buffer-31} 
    $\Gamma\cup \{ \Lis = \Q\comp\F \} \vdash_{\Delta[\Lis,\Q,\Ele,\F]} \PR[\Q\comp\F\comp\Ele]$ & 
    by \Rew~ from the third assertion\\
  
  \label{ps:buffer-32} 
  $\Gamma\cup \{ \Lis = \Q\comp\F \} \vdash_{\Delta[\Lis,\Q,\Ele,\F]} \PR[\Lis\comp\Ele]$ & 
  by \Rew~ and \Trans~ from \ref{ps:buffer-31}\\
  
   $\Gamma\cup \{ \Exists{\Q,\F}\Lis = \Q\comp\F \} \vdash_{\Delta[\Lis,\Ele]} \PR[\Lis\comp\Ele]$ & 
  by \QuantI~ from \ref{ps:buffer-32}
  \endsubsteps
 \label{ps:buffer-4-4}
 $\Gamma\vdash_{\Delta[\Lis,\Ele]} \PR(\Lis\comp\Ele)$  &
 by \DisjE~ from \ref{ps:buffer-4-1}, \ref{ps:buffer-4-2} and \ref{ps:buffer-4-3}\\
 
 $\Gamma\vdash_\Delta \Forall{\Lis,\Ele}\PR(\Lis\comp\Ele)$ &
 
 from \ref{ps:buffer-4-4},
 \newline $\Forall{\Lis,\Ele}\PR[\Lis\comp\Ele]$ is a rigid sentence 
 \end{proofsteps}
 
 The proof of the fifth assertion resembles the proof of the fourth assertion.
 \end{proof}


It is worth noting that the expressivity of the basic layer of $\HFOLR$ allows a simple description of the property to prove $\Forall{\Lis}(@_\lifo\del)((@_\fifo\del)(\Lis))=(@_\fifo\del)((@_\lifo\del)(\Lis))$. 
 The same property can be expressed in $\HFOLS$ as follows:
 $\Forall{\Lis}\Exists{x_1,x_2,y_1,y_2}$ 
 $x_1=y_1 \wedge 
 \at{\lifo} (x_1 = \del(x_2))) \wedge
 \at{\fifo} (x_2 = \del(\Lis)) \wedge
 \at{\fifo} (y_1 = \del(y_2))) \wedge
 \at{\lifo} (y_2= \del(\Lis))$, 
 where $x_i$, $y_i$ are variables of sort $\List$.
It is not difficult to see that the expressivity of $\HFOLR_b$ has deep ramifications in formal verification; for example, the proofs become much simpler.

%%%%%%%%%%%%%%%%%%%%%%%%%%%%%%%%%%%%%%%%%%%%%%%%%%%%%%%%%%%%%%%%%%%%%%%%%%%%%
\appendix

\section{Proof rules}

 %%%%%%%%%%%%%%%%%%%%%%%%%%%%%%%%%%%%%%%%%%%%%%%%%%%%%%%%%%%%%%%%%%%%%%%%%%%%%%%%%%%%%% 
\mytable{tab3}{Proof rules for the basic layer}{
 \small
 \begin{tabular}{|l l|}
  \hline
  
  & \\
  
   \Rnom~\inferrule{}{\Gamma\vdash \at{k} k} \hspace{0.5cm}
  
   \Pnom~\inferrule{ \Gamma\vdash \at{k}\underline\lambda(k_1) \Space \Gamma \vdash \at{k_1} k_1'}{\Gamma \vdash \at{k}\underline\lambda(k_1')} & 
  
   \Cnom~\inferrule{ \Gamma\vdash \at{k_1} e \Space \Gamma\vdash \at{k_1} k } { \Gamma \vdash \at{k} e } \\
  
  & \\
  
   \Rhyb~\inferrule{ }{ \Gamma \vdash t = t } \hspace{0.5cm}

   \Shyb~\inferrule{ \Gamma \vdash t_{1} = t_{2}}{\Gamma \vdash t_{2} = t_{1}} &

   \Thyb~\inferrule{\Gamma \vdash t_{1} = t_{2} \Space \Gamma \vdash t_{2} = t_{3}}{\Gamma \vdash t_{1} = t_{3}} \\
  
  & \\
   
   \Fhyb~\inferrule{\Gamma \vdash t_1 = t_2}{\Gamma \vdash \sigma(t_1) = \sigma(t_2)} [\ $\sigma\in\overline{F}$\ ] & 
   
   \Phyb~\inferrule{\Gamma \vdash t_1 = t_2 \Space \Gamma \vdash \pi(t_1)}{\Gamma \vdash \pi(t_2)} [\ $\pi\in\overline{P}$] \\
   
  & \\
   
   \Whyb~\inferrule{\Gamma \vdash \at{k_1} k \Space \Gamma\vdash \rho}{\Gamma \vdash \varphi_{k\leftarrow k_1}(\rho)} &
   
   \Wrig~\inferrule{ \Gamma\vdash \at{k_1} k }{ \Gamma \vdash \varphi_{k\leftarrow k_1}(t) =_s t }~[\ $s\in S^\rigid$\ ] \\
  
  & \\
  
   \Relim~\inferrule{\Gamma\vdash \at{k}\rho}{\Gamma\vdash\At{k} \rho} &
   \Rintro~\inferrule{\Gamma\vdash\At{k} \rho}{\Gamma\vdash \at{k}\rho}\\
   
  & \\ 
  
 \hline
 \end{tabular}
}
%%%%%%%%%%%%%%%%%%%%%%%%%%%%%%%%%%%%%%%%%%%%%%%%%%%%%%%%%%%%%%%%%%%%%%%%%%%%%%%%%%%%%%

%%%%%%%%%%%%%%%%%%%%%%%%%%%%%%%%%%%%%%%%%%%%%%%%%%%%%%%%%%%%%%%%%%%%%%%%%%%%%%%%%%%%%%
\begin{table}[h]
 \small
%%%
 \begin{tabular}{|l l|}
  
  \hline
  
  & \\

  \Cons~\inferrule{\varphi(\Gamma)\vdash \varphi(e)}{\Gamma\vdash e}~[ \ $\varphi$ is conservative  \ ] &  
  
  \Subst~\inferrule{\Gamma\vdash \at{k}\theta( e') }{\Gamma \vdash \at{k}\Exists{\chi}e'}~[ \ $\theta\colon 1_\chi\to 1_\Delta$ \ ] \\

  
  & \\
  
  \NegI~\inferrule{\Gamma\cup\{\at{k} e \}\vdash \bot}{\Gamma\vdash \at{k} \neg e} & 
  
  \NegE~\inferrule{\Gamma\vdash \at{k} \neg e}{\Gamma\cup\{\at{k} e\}\vdash \bot} \hfill

  \NegD~\inferrule{\Gamma\vdash \at{k}\neg\neg e }{ \Gamma \vdash \at{k} e} \\ 
  
   & \\
  
  \FalseI~\inferrule{\Gamma\vdash  \at{k} e \Space \Gamma\vdash \at{k} \neg e }{ \Gamma \vdash\bot} & 
  
  \FalseE~\inferrule{ \Gamma\vdash \bot }{\Gamma \vdash E} \\ 
  
  & \\
  
  \DisjI~\inferrule{ \Gamma\vdash \at{k} e }{ \Gamma \vdash \at{k} \vee E}~[ \ $e\in E$ \ ] & 
  \DisjE~\inferrule{\Gamma\vdash \at{k} \vee E  \Space \Gamma\cup\{\at{k} e\}\vdash \gamma \mbox{ for all } e\in E }{\Gamma\vdash\gamma}  \\
  
  & \\
  
  \PosI~\inferrule{\chi_z(\Gamma)\cup\{ \at{k}\underline\lambda(z),\at{z}\chi_z(e)\}\vdash \chi_z(\gamma)} { \Gamma \cup \{ \at{k}\pos{\lambda}e \} \vdash \gamma } & 
  \PosE~\inferrule { \Gamma \cup \{ \at{k} \pos{\lambda}e \}\vdash \gamma } {\chi_z(\Gamma)\cup\{ \at{k}\underline\lambda(z),\at{z}\chi_z(e)\}\vdash \chi_z(\gamma)} \\ 
  
  & \\
    
  \QuantI~\inferrule{ \chi(\Gamma)\cup \{ \at{k'} e' \}\vdash \chi(\gamma) } { \Gamma \cup \{ \at{k} \Exists{\chi}e' \}\vdash \gamma }~[ \ $k'=F(\chi)(k)$ \ ] & 
  
  \QuantE~\inferrule{\Gamma \cup \{ \at{k} \Exists{\chi}e' \}\vdash \gamma}{ \chi(\Gamma)\cup \{ \at{k'} e' \}\vdash \chi(\gamma) }~[ \ $k'=F(\chi)(k)$ \ ] \\
  
  & \\
  
 
  \StoreI~\inferrule{ \Gamma\vdash \at{k}\varphi_{z\leftarrow k}(e'') }{ \Gamma \vdash \at{k}\store{z}e''} & 
   
  \StoreE~\inferrule{ \Gamma\vdash \at{k}\store{z} e''  }{ \Gamma \vdash \at{k} \varphi_{z\leftarrow k}(e'')} \\
 
 & \\
 
  \RetI~\inferrule{\Gamma \vdash e }{\Gamma\vdash \at{k} e} & 
  
  \RetE~\inferrule{\chi_z(\Gamma)\vdash @_z \chi_z(e)}{\Gamma\vdash e} \\ 
   
   & \\ 
 \hline
  
\end{tabular}
%%%
\caption {Proof rules for stratified institutions}
\label{tab1}
\end{table}


\end{document}