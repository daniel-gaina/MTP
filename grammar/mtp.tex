\documentclass[10pt,a4paper]{article}
\usepackage{geometry}
\geometry{a4paper}
\geometry{margin=2.5cm,nohead}

%% Use the option review to obtain double line spacing
%% \documentclass[preprint,review,12pt]{elsarticle}

%% Use the options 1p,twocolumn; 3p; 3p,twocolumn; 5p; or 5p,twocolumn
%% for a journal layout:
%% \documentclass[final,1p,times]{elsarticle}
%% \documentclass[final,1p,times,twocolumn]{elsarticle}
%% \documentclass[final,3p,times]{elsarticle}
%% \documentclass[final,3p,times,twocolumn]{elsarticle}
%% \documentclass[final,5p,times]{elsarticle}
%% \documentclass[final,5p,times,twocolumn]{elsarticle}

%% if you use PostScript figures in your article
%% use the graphics package for simple commands
%% \usepackage{graphics}
%% or use the graphicx package for more complicated commands
%% \usepackage{graphicx}
%% or use the epsfig package if you prefer to use the old commands
%% \usepackage{epsfig}

%% The amssymb package provides various useful mathematical symbols
\usepackage{amssymb}
%% The amsthm package provides extended theorem environments
\usepackage{amsthm}
%\usepackage{url}
%\usepackage{alltt}
%\usepackage[english]{babel}
%\usepackage{amssymb}
 \usepackage[final]{hyperref}
\usepackage{hyperref}
\usepackage{color} %para el texto con colores
\usepackage[inference]{semantic}
\usepackage{fixme}
\usepackage{graphics} %para el scalebox
\usepackage[pagewise]{lineno} %Para los numeros de linea
\usepackage{rotating}
\usepackage{multirow}
\usepackage[T1]{fontenc}  % quita el warning OMS/cmtt/m/n y OMS/cmsy/m/n


%\usepackage[margins]{trackchanges}


% Para verbatim en captions
%\usepackage{cprotect}

%\usepackage{proof}

%% The lineno packages adds line numbers. Start line numbering with
%% \begin{linenumbers}, end it with \end{linenumbers}. Or switch it on
%% for the whole article with \linenumbers after \end{frontmatter}.
%% \usepackage{lineno}

%% natbib.sty is loaded by default. However, natbib options can be
%% provided with \biboptions{...} command. Following options are
%% valid:

%%   round  -  round parentheses are used (default)
%%   square -  square brackets are used   [option]
%%   curly  -  curly braces are used      {option}
%%   angle  -  angle brackets are used    <option>
%%   semicolon  -  multiple citations separated by semi-colon
%%   colon  - same as semicolon, an earlier confusion
%%   comma  -  separated by comma
%%   numbers-  selects numerical citations
%%   super  -  numerical citations as superscripts
%%   sort   -  sorts multiple citations according to order in ref. list
%%   sort&compress   -  like sort, but also compresses numerical citations
%%   compress - compresses without sorting
%%
%% \biboptions{comma,round}

% \biboptions{}

\newtheorem{assumption}{Assumption}[section]
\newtheorem{property}{Property}[section]
\newtheorem{example}{Example}[section]
\newtheorem{definition}{Definition}[section]
\newtheorem{lemma}{Lemma}[section]
\newtheorem{theorem}{Theorem}[section]

\newcommand{\codesize}{\small}
\newcommand{\vs}{\vspace{-1.15ex}}

\newcommand{\mi}[1]{\mathit{#1}}
\newcommand{\APT}{\mi{APT}}
\newcommand{\APTZ}{\mi{APTZ}}
\newcommand{\vals}{\mi{vals}}
\newcommand{\vars}{\mi{vars}}
\newcommand{\val}{\mi{val}}
\newcommand{\exprs}{\mi{exprs}}
\newcommand{\pats}{\mi{pats}}
\newcommand{\fails}{\mi{fails}}
\newcommand{\succeeds}{\mi{succeeds}}
\newcommand{\evals}{\mi{evals}}
\newcommand{\eval}{\mi{eval}}

\newcommand{\lrangle}[2]{\langle #1, #2\rangle}
\newcommand{\ojo}[1]{\textbf{\textcolor{red}{#1}}}
\newcommand{\sub}[1]{\lrangle{#1}{\theta}}
\newcommand{\subprime}[1]{\lrangle{#1}{\theta'}}
\newcommand{\subid}[1]{\lrangle{#1}{\mi{id}}}
\newcommand{\subinit}[1]{\lrangle{#1}{\mi{id}}}
\newcommand{\subi}[2]{\lrangle{#1}{\theta_{#2}}}
\newcommand{\subdp}[1]{\lrangle{#1}{\theta''}}

\newcommand{\lab}[1]{\mathsf{(#1)}}
\newcommand{\labapt}[1]{\!{\scriptstyle\lab{#1}}}
\newcommand{\labtxt}[1]{$\mathsf{(#1)}$}

\newcommand{\CESC}{\mi{CESC}} % usage:  $\CESC$
\newcommand{\CESCI}[1]{\mi{ICESC_{#1}}} % usage:  $\CESCI{Z}$
\newcommand{\CESCZ}{\mi{ZESC}} % usage:  $\CESCZ$
\newcommand{\zoom}{\mathcal{Z}} % usage:  $\CESCZ$

\makeatletter
\newenvironment{prog}{\vspace{1.0ex}\par
\obeylines\@vobeyspaces\tt}{\vspace{1.0ex}\noindent } \makeatother
\newcommand{\startprog}{\begin{prog}}
\newcommand{\stopprog}{\end{prog}\noindent}

\newcommand{\lin}[1]{line~\textsf{#1}}
\newcommand{\linann}[1]{\makebox[1em][r]{\textsf{\scriptsize #1}}}

% Para reducciones
\newcommand{\reduc}{\mathcal{E}}

% Blah blah
\newcommand{\blah}{\ojo{Blah Blah}}

\title{MTP Grammar
%\thanks{Research supported by MICINN Spanish project
%\emph{StrongSoft} (TIN2012-39391-C04-04)
%and Comunidad de Madrid program \emph{PROMETIDOS} (S2009/TIC-1465).}
}

\author{TBA
%Rafael Caballero, Enrique Martin-Martin, Adri\'an Riesco, and Salvador Tamarit\\[.7cm]
%\normalsize Technical Report 01/14\\[1ex]
%  \normalsize\textit{Departamento de Sistemas Inform\'aticos y Computaci\'on},\\
%  \normalsize\textit{Universidad Complutense de Madrid}\\[.4cm]
%  June, 2014\\
%  (updated on \today)
  }

\date{}

\begin{document}

\maketitle



\title{}

%% use optional labels to link authors explicitly to addresses:
%% \author[label1,label2]{<author name>}
%% \address[label1]{<address>}
%% \address[label2]{<address>}

\begin{abstract}
%% Text of abstract
TBA
\smallskip

\noindent\textbf{Keywords:} TBA
\end{abstract}


%% main text

\section{MTP Grammar\label{sec:grammar}}
%!TEX root = mtp.tex


\begin{figure}[t]
\scalebox{.89}{
\begin{tabular}{lll}
Spec   & ::= & spec ME \texttt{=} Decs \\
ME     & ::= & TId \textbar~TId \texttt{\{} Params \texttt{\}}\\
Params & ::= & PId \texttt{::} TId \textbar~$\varepsilon$
%lit   & ::= & Atom \textbar~Integer \textbar~Float \textbar~Char \textbar~String %\textbar~ BitString
%\textbar~\texttt{[~]}\\
%fun   & ::= &  \texttt{fun(}var$_1$ \texttt{,} \ldots \texttt{,} var$_n$) \texttt{->} exprs \\
%clause & ::= & pats \texttt{when} exprs$_1$ \texttt{->} exprs$_2$\\
%pat   & ::= & var \textbar~lit \textbar~\texttt{[} pats$_1$ \texttt{\textbar} pats$_2$ \texttt{]} \textbar~\texttt{\{} pats$_1$\texttt{,} \ldots \texttt{,} pats$_n$ \texttt{\}} \textbar~var \texttt{=} pats \\
%pats  & ::= & pat \textbar~\texttt{<}\ pat$_1$\texttt{,} \ldots \texttt{,} pat$_n$\ \texttt{>} \\
%exprs & ::= & expr~\textbar~\texttt{<}\ expr$_1$\texttt{,} \ldots \texttt{,} expr$_n$\ \texttt{>} \\
%expr  & ::= & var \textbar~lit \textbar~$\xi$ \textbar~fname \textbar~fun \textbar~\texttt{[} exprs$_1$ \texttt{\textbar} exprs$_2$ \texttt{]} \textbar~\texttt{\{} exprs$_1$\texttt{,} \ldots \texttt{,} exprs$_n$ \texttt{\}} \\
%      &     & \textbar~\texttt{let} vars \texttt{=} exprs$_1$ \texttt{in} exprs$_2$ \\
%      &     & \textbar~\texttt{letrec} fname$_1$ \texttt{=} fun$_1$ \ldots fname$_n$ \texttt{=} fun$_n$ \texttt{in} exprs \\
%      &     & \textbar~\texttt{apply} exprs$_{n+1}$ \texttt{(} exprs$_1$ \texttt{,} \ldots \texttt{,} exprs$_n$ \texttt{)}\\
%      &     & \textbar~\texttt{call} exprs$_{n+1}$\texttt{:}exprs$_{n+2}$ \texttt{(} exprs$_1$ \texttt{,} \ldots \texttt{,} exprs$_n$ \texttt{)}\\
%      &     & \textbar~\texttt{primop} Atom \texttt{(} exprs$_1$ \texttt{,} \ldots \texttt{,} exprs$_n$ \texttt{)}\\
%      &     & \textbar~\texttt{try} exprs$_1$ \texttt{of} \texttt{<}var$_1$ \texttt{,} \ldots \texttt{,} var$_n$\texttt{>}~\texttt{->} exprs$_2$\\
%      &     & ~~\texttt{catch} \texttt{<}var'$_{1}$ \texttt{,} \ldots \texttt{,} var'$_{m}$\texttt{>}~\texttt{->} exprs$_3$\\
%      &     & \textbar~\texttt{case} exprs \texttt{of} clause$_1$ \ldots clause$_n$ \texttt{end}\\
%      &     & \textbar~\texttt{do} exprs$_1$ exprs$_2$ \textbar~ \texttt{catch} exprs\\
%$\xi$ & ::= & Exception(val$_1$, \ldots, val$_m$) \\
%val   & ::= & lit \textbar~fname \textbar~fun \textbar~\texttt{[} vals$_1$ \texttt{\textbar} vals$_2$ \texttt{]} \textbar~\texttt{\{}vals$_1$\texttt{,} \ldots \texttt{,} vals$_n$\texttt{\}}\\
%eval   & ::= & lit \textbar~fname \textbar~fun \textbar~\texttt{[} evals$_1$ \texttt{\textbar} evals$_2$ \texttt{]} \textbar~\texttt{\{}evals$_1$\texttt{,} \ldots \texttt{,} evals$_n$\texttt{\}} \textbar~$\xi$\\
%vals  & ::= & val \textbar~\texttt{<}\ val$_1$\texttt{,} \ldots \texttt{,} val$_n$\ \texttt{>} \\
%evals  & ::= & eval \textbar~\texttt{<}\ eval$_1$\texttt{,} \ldots \texttt{,} eval$_n$\ \texttt{>} \\
%vars  & ::= & var \textbar~\texttt{<}\ var$_1$\texttt{,} \ldots \texttt{,} var$_n$\ \texttt{>} \\
\end{tabular}
}
\caption{MTP's Syntax}
\label{fig:syntax}
\end{figure}

We have the following categories:
\begin{itemize}
\item
Spec, for a theory.

\item
ME, for module expressions.

\item
Decs, for the declarations in the theory.

\item
TId, for theory identifiers.

\item
PId, for parameter identifiers.

\item
Params, for defining theory parameters.
\end{itemize}






















%% \section{}
%% \label{}

%% References
%%
%% Following citation commands can be used in the body text:
%% Usage of \cite is as follows:
%%   \cite{key}         ==>>  [#]
%%   \cite[chap. 2]{key} ==>> [#, chap. 2]
%%

%% References with bibTeX database:

{
\bibliographystyle{plain}
\bibliography{../refs.bib}
}

%% Authors are advised to submit their bibtex database files. They are
%% requested to list a bibtex style file in the manuscript if they do
%% not want to use elsarticle-num.bst.

%% References without bibTeX database:

% \begin{thebibliography}{00}

%% \bibitem must have the following form:
%%   \bibitem{key}...
%%

% \bibitem{}

% \end{thebibliography}

%% The Appendices part is started with the command \appendix;
%% appendix sections are then done as normal sections

\end{document}

%%
%% End of file `elsarticle-template-num.tex'.
