%!TEX root = mtp.tex


\begin{figure}[t]
\scalebox{.89}{
\begin{tabular}{lll}
Spec    & ::= & \texttt{spec*} ME \texttt{=} Decs \texttt{end} \textbar\\
        &     & \texttt{spec!} ME \texttt{=} Decs \texttt{end} \\
ME      & ::= & Id \textbar~Id \texttt{\{} Params \texttt{\}}\\
Params  & ::= & Id \texttt{::} Id, Params \textbar~$\varepsilon$\\
Decs    & ::= & Dec\ Decs \textbar~$\varepsilon$\\
Dec     & ::= & ImpDecl\ \textbar~SortDecl\\
ImpDecl & ::= & \texttt{pr} \\
\end{tabular}
}
\caption{MTP's Syntax}
\label{fig:syntax}
\end{figure}

{\color{red}Question:} we talked about how to define sorts, but I do not remember because
we changed a couple of times. I think we said we should have a single sort keyword (\texttt{sort},
with the synonym \texttt{sorts} maybe) because the same sort might behave in one way for some
modules and in some other for other modules, do I remember correctly?

We have the following categories:
\begin{itemize}
\item
Spec, for a theory.

\item
ME, for module expressions.

\item
Decs, for the declarations in the theory.

\item
Id, for identifiers (tokens).

\item
IdL, for lists of identifiers (tokens).

\item
Params, for defining theory parameters.

\item
ImpDecl, for importation declarations.

\item
SortDecl, for sort declarations.
\end{itemize}



















